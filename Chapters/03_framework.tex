\chapter{The framework}
\label{chap:framework}

...

\section{Requirements}

\subsection{Goals and functional requirements}

The overreaching goal of the framework is to provide a tool for the evaluation of neuroevolution algorithms on benchmark problems.
Tests are specified through a command line interface, they consist in an algorithm, problem pair and a set of additional parameters.
The framework collects the results of the tests as well as the passed-in parameters.
Tests can either be run individually or in batch mode, where the framework runs a set of tests in parallel, and collects statistics on these runs.

Furthermore, the framework allows for the visualization of the problem, solution process and network structure through a graphical user interface.
In addition, it generates graphs for the tests, showing the performance of the algorithm on the problem over the generation count.

\subsection{Non-functional requirements}

Non-functional requirements are the requirements that specify the quality of the system, rather than the features it should have.
Apart from the functional requirements that specify the features expected of the framework, a number of non-functional requirements have also been identified.

\begin{itemize}
    \item Usability and user experience: the framework should be easy to use and provide a good user experience.
    \item Documentation: The framework should be well documented, providing a clear and concise guide on how to use it.
    \item Error handling: All errors should be handled gracefully as to not result in runtime errors.
    \item Performance: The framework should allow for the execution of tests in parallel, making use of multiple CPU cores.
    \item:Extensibility: The framework should be easily extensible, allowing for the addition of algorithms and benchmarks without any
    major changes to the existing codebase.
    \item Support: The framework should be able to run on the three major operating systems: Windows, Linux and MacOS.
\end{itemize}
