\chapter{Literature Review}
\label{chap:review}

...

\section{Neuroevolution algorithms}

\subsection{(1 + 1) NA}

The (1 + 1) NA algorithm and its variants were introduced in \cite{na}.
In this work, the authors consider a simple neuroevolution setting where these algorithms are used to optimize the weights and activation function of
a simple artificial neural network.

\subsubsection{The artificial neural network}

Artificial neurons with $D$ inputs and a binary threshold activation function are considered.
These neurons have $D$ parameters, the input weights $w_1, \ldots, w_D$ and the threshold $t$.
Let $x = (x_1, \ldots, x_D) \in \mathds{R}^D$ be the input of the neuron. The neuron outputs $1$ if $\sum_{i=1}^D w_i x_i \geq t$ and $0$
otherwise.
This can be interpreted geometrically as the neuron outputting $1$ if the input vector $x$ is above or on the hyperplane with normal vector
$w = (w_1, \ldots, w_D)$ and bias $t$.
j
Furthermore, an alternative representation of the decision hyperplane can be used by considering spherical coordinates.
The normal vector to the decision hyperplane is described by by $D - 1$ angles and the bias.
As a matter of fact, for $D = 2$, the normal vector can be represented by its cartesian coordinates $(x_1, x_2)$ or by its polar coordinates
$(r, \theta)$, where $r$ is the distance from the origin and $\theta$ is the angle with the $x_1$ axis. Similarly, for $D = 3$, the normal vector can
be represented by its cartesian coordinates $(x_1, x_2, x_3)$ or by its spherical coordinates $(r, \theta, \phi)$, where $r$ is the distance from the
origin, $\theta$ is the angle with the $x_1$ axis and $\phi$ is the angle with the $x_3$ axis.
It is easy to convert between these two representations. In addition, the spherical representation uses one less parameter than the cartesian
representation, and hence, allows for the reduction of the number of inputs to the neurons to $D - 1$.

% TODO coordinates figure

The ANNs which are considered in the study contain two layers, a hidden layer with $N > 1$ neurons and an output layer with a single neuron.
Each of the hidden neurons are connected to the $D$ inputs and output a binary value. The output neuron is connected to the $N$ hidden neurons and
computes the Boolean OR function of their outputs.
This architecture is motivated by the problems which are considered in the study, described in \Cref{subsec:na-problems}.
Geometrically, these ANNs output the union of a number of $N$-dimensional hyperplanes.

% TODO ANN figure

\subsubsection{The (1 + 1) NA algorithm}

Let's consider an ANN with $N$ neurons in the hidden layer and $D$ inputs, with parameters
$(\phi_{1,1}, \dots, \phi_{1,D-1}, b_1, \dots, \phi_{N,1}, \dots, \phi_{N,D-1}, b_N$.
In the paper \cite{na}, the search space $[0 , \dots, n]^{N D}$ is considered, where $r$ is the resolution of the continuous $[0, 1]$ domain.
This discretiisation allows for the values $\{0, 1/r, 2/r, \dots, 1\}$. Setting the parameters of ANNs is tipically a continous optimization problem,
but rigurous runtime analysis is much less developed for continous optimization than for discrete optimization, which motivates this choice.
Let $f : \{0 , \dots, r\}^{N D} \to [0, 1]$ be the fitness function which measures the performance of the ANN and is to be maximized.

The (1 + 1) NA algorithm is given in \Cref{alg:na}.
It maintains a single individual and mutates all angles and biases independently, based on a global search operator using the harmonic distribution
$\text{Harm}(r)$ on $\{1, \dots, r\}$: For $l \sim \text{Harm}(r)$,

\[
    Prob(l = i) = \frac{1}{H_r} \text{ for } i = 1, \dots, r, \text{ where } H_r = \sum_{i=1}^r \frac{1}{i}.
\]

\begin{algorithm}
\caption{(1 + 1) NA}
\label{alg:na}
\begin{algorithmic}
    \State $t \gets 0$
    \State Select $x_0$ uniformly at random from $\{0, \ldots, r\}^{DN}$.
    \While{termination criterion not met}
        \State Let $y = (\varphi_{1,1}, \ldots, \varphi_{1,D-1}, b_1, \ldots, \varphi_{N,1}, \ldots, \varphi_{N,N-1}, b_N) \gets x_t;$
        \ForAll{$i \in \{1, \ldots, N\}$}
            \State Mutate $\varphi_i$ and $b_i$ with probability $\frac{1}{DN}$, independently of each other and other indices;
            \State Mutation chooses $\sigma \in \{-1, 1\}$ uniformly at random and $l \sym \text{Harm}(r)$ and adds $\sigma l$ to the selected component, the
            result is then taken modulo $r$ for angles and modulo $r + 1$ for biases.
            \For {$i \in \{1, \ldots, N\}$}
                \State Set bias $2b_i / r - 1$ for neuron $i$.
                \For {$j \in \{1, \ldots, D\}$}
                    \State Set the $j$-th polar angle to $2\pi \varphi_{i,j} / r$ for neuron $i$.
                \EndFor
            \EndFor
            \State Evaluate $f(y)$
            \If{$f(y) \geq f(x_t)$}
                \State $x_{t+1} \gets y$
            \Else
                \State $x_{t+1} \gets x_t$
            \EndIf
        \EndFor
        \State $t \gets t + 1$
    \EndWhile
\end{algorithmic}
\end{algorithm}

\subsection{Bias-Invariant (1+1) NA (BINA)}

...

\section{Neuroevolution benchmarks}

...
