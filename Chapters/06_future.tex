\chapter{Future work}

% TODO refer to previous parts of the report somehow

Neuroevolution is a particularly vast field and this five month project has only scratched the surface of the evaluation of neuroevolution algorithms. Hence, many areas of future work are possible,

First of all, this project's literature review, algorithm implementation and evaluation focussed on a limited number of neuroevolution algorithms out of the many that exist. Future work could involve
the implementation and evaluation of other algorithms, including potential new ones that may be developed in the future. This would result in a more comprehensive evaluation of neuroevolution algorithms.

Similarly, the evaluation of the algorithms could be extended to include more benchmarks. For example, considering other control benchmarks or more complex binary classification datasets.
This would allow to further validate the results obtained in this project and to provide a more comprehensive evaluation of the algorithms.

Furthermore, more experiments on the selected benchmarks could be conducted, such as tuning the parameters of the CMA-ES algorithm, or expanding the network architectures by considering more
hidden layers, allowing for recurrent connections or experimenting with other activation functions. Furthermore regarding the $(1 + 1)$ NA and BNA algorithm, future work possibilities mentioned
in \cite{na} and \cite{bna} could also be considered, such as experimenting with continuous versions of the algorithms (which are actually already implemented in the framework for the unit tests).

Moreover, the evaluation process could also be extended to include more metrics. For example, adding the number of fitness evaluations which could be more relevant when comparing different algorithm
than the number of iterations. In addition, computing the percentage of optimal solutions found by the algorithms could also be interesting to consider.

Finally, future work could also involve improvements and extensions to the framework. For example, adding support for serialization of the evolved networks and loading them back for testing, making
use of parallelism in the algorithms themselves, which is already supported by the \texttt{cmaes} crate but was not used for this project to keep the results comparable with the manually implemented
algorithms, or adding support for extracting the evolved network parameters to readable text files for further analysis, which could be particularly in the case of the NEAT algorithm.
