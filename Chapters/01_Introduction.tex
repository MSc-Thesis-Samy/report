\chapter{Introduction}

Neuroevolution is a subfield of artificial intelligence which consists in the evolution of ANNs (artificial neural networks).
ANNs are traditionally trained using gradient-based methods, such as stochastic gradient descent.
Over the years, these methods have been successfully applied to a variety of problems, such as image classification, speech recognition and natural language processing.
Such problems allow for supervised learning, where ANNs are trained on a dataset of input-output pairs.
However, there is a class of problems for which supervised learning is not applicable, where instead of input-output pairs, only a measure of performance is available.
In addition, the performance of ANNs is also heavily impacted by their architectures. The design of ANNs architecture is a complex and time-consuming task, which is
topically done by hand, based on experience.

Neuroevolution, on the other hand, as a more general approach, can in particular be applied to this other class of problems, as well as to the design of ANNs architectures.
This method is based on evolutionary algorithms, which are inspired by the process of natural selection. These algorithms maintain a population of individuals,
which are mutated and recombined to evolve towards optimal solutions. They have shown success for black-box problems and have successfully been applied to a wide range
of engineering problems.

The field of neuroevolution has been researched for over 40 years, hence many different algorithms, benchmarks and applications have been proposed.

...

\subsection{Overview}

TODO
