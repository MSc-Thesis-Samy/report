\chapter{Background}
\label{chap:background}

\section{Artificial Neural Networks}

Artificial neural networks (ANNs) are a class of machine learning models, which are inspired by biological neural networks. ANNs are composed of interconnected
neurons, which are organized in layers. The first layer of an ANN is referred to as the input layer, the last layer is the output layer, and the layers
in between are hidden layers. In feed-forward ANNs, nodes in a layer are connected to nodes from the immediately preceding and succeeding layers. The connections between nodes
are associated weights.  Signal travels from the input layer to the output layer and the output of a node is computed by applying a non-linear activation function
to the weighted sum of the inputs. The weights of the connections are typically learned using gradient-based optimization algorithms, such as backpropagation.
The performance of an ANN is evaluated using a loss function, which measures the difference between the predicted output and the true output.

% TODO figure + equation

Any function can be approximated by ANNs. Over the years, these models have been applied to a wide range of problems, including classification and reinforcement
learning tasks. The process of applying ANNs consists in the design of its architecture, the choice of parameters such as the loss function or parameters for the
weigh optimization algorithm, and the optimization of its weight, in order to maximize its performance on a given task.

Many challenges are associated with the use of ANNs, such as the choice of the network architecture and hyperparameters, which can have an significant
impact on the performace of the model. Furthermore, the optimization of the network weights using gradient=based algorithms can result in various issues, such
as overfitting, which arises when the model performs well on the training data but poorly on unseen data, suboptimal local minima, and vanishing and
exploding gradients, when using deep networks.

\section{Evolutionary Algorithms}

Evolutionary algorithms (EAs) are a class of optimization algorithms, which are inspired by the process of natural selection. EAs are based on the idea of
evolving a population of candidate solutions to a problem, in order to find the best solution. The process of evolution consists in the selection of the
fittest individuals, which are then recombined to produce offspring, which are then mutated. The fitness of the offspring is evaluated and the process is
repeated for a number of generations until a stopping criterion is met, such as a maximum number of generations or a desired level of performance.
At each generation, the population is updated by replacing the worst-performing individuals with the offspring.

Individuals are represented by genotypes, which are encoded in a way that allows the application of genetic operators, such as crossover and mutation, and
by phenotypes, which are the actual interpretation of the genotype used as solutions to the problem. The fitness of an individual is evaluated using a
fitness function, which measures the quality of the solution.

Mutations are used to introduce diversity in the population, which is important to avoid premature convergence to suboptimal solutions. During mutations,
parts of the genotype are modified while other parts remain unchanged, with the goal of performing a local search around the current genotype.
The individuals reproduce using crossover, which combines the genotypes of two parents to produce offspring. The offspring inherit parts of the genotypes
of the parents. The idea is that combining two good solutions can produce an even better solution.

%B TODO Discuss the different kinds of EAs?

An issue which can arise when using EAs is the stagnation of the population, which occurs when the population converges to a suboptimal solution and
no further improvement is made. In addition, the choice of multiple parameters, such as the mutation rate, the crossover rate, and the population size
can have a significant impact on the performance of the algorithm. Lastly, there is no guarantee that EAs will find the optimal solution to a problem and
not be stuck in a local optima.

\section{NeuroEvolution}

Neuroevolution is a subfield of machine learning, which combines evolutionary algorithms and artificial neural networks. The idea is to evolve the weights,
and potentially, the architecture or hyperparameters of ANNs using an EA instead of traditional gradient-based methods.

...
